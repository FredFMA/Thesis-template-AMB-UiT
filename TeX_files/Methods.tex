\setlength{\parindent}{2em} 
\section{Methods}
\subsection{Methods}

%example of simple table:
\begin{table}[h]
	\centering
	\caption{\footnotesize Example table.}
	\label{tab:Dilutions}
	\footnotesize
	\begin{tabular}{c|ccc}
		\\
		Slide & Vim V9 (1:x, ug/ml) & Alexa 647 (1:x, ug/ml) & Alexa 488 (1:x, ug/ml) \\ \hline
		1 & 1:10 000 , 0.1 & 1:20 000, 0.1 & \\
		2 & 1:2000 , 0.5 & " & \\
		3 & 1;1000 , 1 & " & \\
		4 & 1:500, 2 & " & \\ \hline
		5 & 1:10 000 , 0.1 & 1:10 000 , 0.2 & \\
		6 & 1:2000 , 0.5 & " & \\
		7 & 1;1000 , 1 & " & \\
		8 & 1:500, 2 & " & \\ \hline
		9 & 1:10 000 , 0.1 & 1:2 000 , 1 & \\
		10 & 1:2000 , 0.5 & " & \\
		11 & 1;1000 , 1 & " & \\
		12 & 1:500, 2 & " & \\ \hline
		13 & 1:10 000 , 0.1 & 1:1 000 , 2 & \\
		14 & 1:2000 , 0.5 & " & \\
		15 & 1;1000 , 1 & " & \\
		16 & 1:500, 2 & " & \\ \hline
		17 & 1:10 000 , 0.1 & 1:500, 4 & \\
		18 & 1:2000 , 0.5 & " & \\
		19 & 1;1000 , 1 & " & \\
		20 & 1:500, 2 & " & \\ \hline
		21 & & & 1:200, 5 \\
		22 & & & 1:500, 2 \\
		23 & & & 1:1000, 1 \\
		24 & & & 1:2000, 0.5
	\end{tabular}
\end{table}
